\documentclass{article}
\usepackage[utf8]{inputenc}

\usepackage[a4paper,
    top=3cm,
    bottom=3cm,
    outer=5cm,
    inner=2cm,
    heightrounded,
    marginparwidth=3cm,
    marginparsep=1cm]{geometry}

\usepackage[dutch]{babel}
\usepackage[colorlinks]{hyperref}
\usepackage{microtype}
\usepackage[hyperref]{ntheorem}
\usepackage{amsmath,amsfonts,amssymb}
\usepackage{parskip}

% Margin notes
\usepackage{marginnote}
\usepackage{mparhack}
\usepackage{marginfix}

\newcommand{\annotation}[1]{%
    \marginpar{\small\textit{#1}}}

% Parskip
\setlength{\parindent}{0pt}

% Title Page
\title{Lessen voor de XXIe eeuw}
\date{\today}
\author{Jonas Devlieghere}

% Q&A
\theoremstyle{plain}
\newtheorem{question}{Vraag}

\theoremstyle{nonumberplain}
\theorembodyfont{}
\newtheorem{answer}{Antwoord}


\begin{document}
\maketitle
\tableofcontents
\newpage

\section{Criminaliteit van de jeugd van tegenwoordig}

\begin{question}
In het debat over jongeren en criminaliteit wordt wel eens gesproken over
zogenaamde ``populaire mythen''. Kies één daarvan en geef - zo nodig op
genuanceerde wijze - aan of die stelling op basis van de beschikbare gegevens
kan worden bevestigd of weerlegd?
\end{question}

\begin{answer}

De eerste ``populaire mythe'' luidt dat de jeugdcriminaliteit stijgt. Om hierop
een antwoord te geven zullen we ons baseren op zowel de ``officiële
statistieken'' als vaststellingen gedaan aan de hand van surveyonderzoek.

De cijfers omtrent de instroom van MOF-dossiers als unieke minderjarigen bij het
OM tonen dat het aantal unieke minderjarigen daalt. Dit loopt tot op zekere
hoogte gelijk met de criminaliteit bij meerderjarigen geregistreerd door de
federale politie. Het aantal jongeren dat een als misdrijf omschreven feit
pleegde en daarvoor in contact kwam met het jeugdparket is aanzienlijk gedaald.
Deze vaststelling sluit aan bij de internationale literatuur en de daarin
beschreven crime drop. De eerste mythe vindt dus geen ondersteuning in deze
cijfers. Dit dient uiteraard genuanceerd te worden daar slecht een fractie van
de reële omvang van de jeugdcriminaliteit  ook geregistreerd wordt.

De zelfrapportagecijfers, gebaseerd op de JOP-monitors tonen een gelijkaardig
verloop. Er is eveneens een lichte stijging waar te nemen in de periode
2005-2006, gevolgd door een afname tussen 2008 en 2013. Ook op basis van deze
gegevens kunnen we niet besluiten dat de jeugdcriminaliteit stijgt. Ook deze
vaststelling dient genuanceerd te worden daar het hier gaat om een steekproef
met alle mogelijke fouten van dien.

We concluderen dat de cijfers nauwelijks of geen ondersteuning bieden voor deze
``populaire mythe''.

\end{answer}

\section{Intelligente Energienetten}

\begin{question}
Bespreek de invloed van variabiliteit, voorspelbaarheid en stuurbaarheid van
hernieuwbare energiebronnen op het evenwicht van vraag en aanbod, in een
intelligent energienet.
\end{question}

\begin{answer}

De elektriciteitsopwekking uit hernieuwbare bronnen is erg variabel, moeilijk
voorspelbaar en amper stuurbaar. Daarom krijgen deze bronnen prioriteit wat wil
zeggen dat deze niet worden afgeschakeld bij een overaanbod. Het is daarom nodig
dat de variabiliteit en sturing van de aanbodzijde naar de vraagzijde wordt
gepropageerd. Dit wil zeggen dat de vraag gestuurd wordt (demand side
management) om deze zo goed mogelijk te laten overeenkomen met de opgewekte
elektriciteit. Bovendien zou het gebruik van slimme meters het mogelijk maken om
meer gedetailleerde profielen op te stellen wat toelaat de vraag en het aanbod
beter op elkaar af te stellen.

\end{answer}

\section{Fiscaliteit voor een vergrijsde samenleving}

\begin{question}
Tax shift versus tax lift en de plaats die de belasting op vermogensmeerwaarden
in dit debat inneemt. Bespreek.
\end{question}

\begin{answer}

De tax shift impliceert een vermindering van de fiscale druk op
arbeidsinkomsten, met deze last te spreiden over alle generaties. De last zou
verlegd kunnen worden naar een taxatie van gerealiseerde meerwaarden. Het gaat
hier dan om zowel vermogenswinst op roerend \annotation{\textbf{Roerend
vermogen}: dividenden, interesten, aandelen,\ldots} en onroerend
vermogen \annotation{\textbf{Onroerend vermogen}: Huuropbrengsten, verkoop van
woningen en gronden}. Het probleem is echter dat, ongeacht of men speculatieve
meerwaarde viseert of niet, de opbrengst ervan moeilijk voorspelbaar is.
Bovendien zou deze heffing weinig tot niets opleveren in periodes van negatieve
koersevolutie. Door hiervoor te kiezen vervangt met een zeker en vast rendement
met een heffing die in bepaalde jaren nauwelijks iets zal opbrengen.

\end{answer}

\section{Het financiële systeem}

\begin{question}
Hebben we het financiële systeem voldoende hervormd volgens u? Neem een
duidelijke positie in en beargumenteer uw positie beknopt door gebruik te maken
van lesmateriaal of ander materiaal in de media.
\end{question}

\begin{answer}

Hoewel er belangrijke maatregels zijn getroffen ben ik van mening dat deze
onvoldoende zijn omdat radicale hervormingen gemeden zijn en mijn inziens
noodzakelijk zijn.

Allereerst zijn er grotere buffers nodig. Ze zijn immers moeilijk te meten,
makkelijker te omzeilen en bijzonder gering in vergelijking met andere sectoren.
De lage risicoweging leidt tot het excessief opstapelen van activa wat
systeemrisico's in de hand werkt. Deze risico's kunnen bovendien niet
betrouwbaar gemeten of geïnternaliseerd worden en onze kennis er omtrent is
beperkt.

Daarnaast worden de publieke vangnetten al maar groter wat risiconame
aanmoedigt. Afwikkelingsmechanismen zijn nog steeds ontoereikend om hier
gestaagd vanaf te stappen.

Tenslotte blijft de geloofwaardigheid in het financieel systeem uit. Er komen
nog steeds schandalen naar boven. Het illustreert dat de ethiek en integriteit
van de grootbanken nog steeds tekortschiet.

\end{answer}

\section{Terugbetaling van geneesmiddelen}

\begin{question}
Licht toe welke redenen aan de basis kunnen liggen waarom de terugbetaling van
geneesmiddelen nog geen Europees gebeuren is.
\end{question}

\begin{answer}

\end{answer}

\section{Secularisatie}

\begin{question}
Leg uit: voor Marcel Gauchet is de ‘dynamiek van transcendentie’ de motor van de
ontwikkeling in de richting van secularisatie.
\end{question}

\begin{answer}

Met `dynamiek van trancendentie' bedoelt Gauchet dat aanvankelijk zowel de
transcendente wereld verstrengeld is met onze concrete, zichtbare wereld. Voor
alles heb je een god (God van de regen, God van de zon, \dots). Met de overgang
van het polythe\"isme naar het monothe\"isme wordt de afstand tussen de twee
werelden groter. Het concept van de monothe\"istische God is al veel abstracter
en de verwevenheid met de concrete wereld een stuk kleiner. Het Christendom
wordt als de voorlaatste fase gezien, met de verlichting als de laatste fase.
Bij de laatste fase is de afstand tussen het transcendente en het concrete het
grootst. Hoe verder we in deze dynamiek van transcendentie gaan, hoe minder
verweven onze concrete wereld is met de transcendente wereld, dus meer seculair.

\end{answer}

\section{De dood van de literatuur?}

\begin{question}
Het artikel is gebaseerd op de overtuiging dat ons moderne ‘concept’ van
literatuur samenhangt met onder meer de verschuiving van een orale naar een
scripturale (of schriftelijke) literatuur. Wat vind je van deze stelling, in het
licht dat er vandaag nog steeds een grote voorleescultuur bestaat (bij jonge
kinderen), dat tal van auteurs voorlezen uit eigen werk voor een groot publiek,
dat sommige jonge dichters hun werk uitdrukkelijk concipiëren voor de mondelinge
voordracht (performerdichters en slammers)? Zijn dit weerleggingen van de
algemene stelling?
\end{question}

\begin{answer}

In elk van de aangehaalde voorbeelden gaat het nog steeds om geschreven teksten
die mondeling worden gedeeld. De ingrijpende veranderingen die de boekdrukkunst
met zich meebracht gelden nog steeds voor deze werken. Er is spraken van een
auteur aan wie het werk toegeschreven kan worden, de auteur van het voorgelezen
werk of de dichter. Het werk overleeft zelfs wanneer het niet mondeling
overgebracht kan worden en het is toegankelijk in zijn tekstuele vorm. Hoewel
zij het orale beogen, genieten deze werken nog steeds van de vele cruciale
kenmerken die het scripturale te bieden heeft. In die zin weerleggen zij niet de
algemene stelling.

\end{answer}

\section{Bouwen is nog geen bouwkunst}

\begin{question}
Het afgebroken paviljoen ontworpen door Toyo Ito in Brugge is meer dan een
constructie opgetrokken in aluminium. Het ontwerp bevat een grote gelaagdheid
die met een minimum aan middelen wordt vertaald. Geef enige toelichting waarom
dit werk van Ito een betekenis heeft.
\end{question}

\begin{answer}

Het ontwerp van Toyo Ito was een U-vormige ruimte, open aan twee zijden en
geplaatst in een ondiepe cirkelvormige vijver. Het was een eerbetoon aan Jan Van
Eyck, die daar begrapen was. Hij was de kunstenaar die op revolutionaire wijze
het daglicht vasthield in zijn werken. De weerkaatsing van het daglicht op het
water van de vijver gaf een wonderlijke reflectie in het interieur van het
paviljoen. De constructie was er zuiver om het daglicht zichtbaar te maken.
Voorheen stond op deze locatie de Sint Donaatskerk waarin een centrale bouw werd
gecombineerd met een lang schip. De basisvorm van het paviljoen verwijst naar de
tweeledigheid van de Romaanse kerk.

\end{answer}

\section{Leren van Tissergate}

\begin{question}
Wat is een `fysieke context' en wat is een `humane context' en hoe kunnen
die toegepast worden op één van de drie projecten BsS, Avalon of B@L?
\end{question}

\begin{answer}

De \emph{context} is het betekenisgevend kader waarbinnen iets plaatsvindt. De
\emph{fysieke context} heeft betrekking op de specifieke architectuur. Hoe is
het project opgetrokken? Voor B@L omvat dit de smalle site zo optimaal mogelijk
benutten, met optimale openheid en lichtinval. Het gaat dan om de opdeling van
de ruimtes tot het kiezen van de juiste materialen in harmonie met de omgeving.

De \emph{humane context} focust zich op het menselijke, sociale en
maatschappelijke aspect. Het tracht dynamiek enthousiasme op gang te brengen.
Voor het B@L project gaat het erom om de verhoudingen tussen publiek en privaat
en tussen jong en oud naar voren te schuiven. De architect dienst hier niet
alleen rekening mee te houden, maar moet deze elementen samenbrengen.

\end{answer}

\section{Toerisme en erfgoed}

\begin{question}
Is Venetië inderdaad een exemplarisch voorbeeld van de toeristische `tragedy of
the commons'? Leg uw antwoord uit.
\end{question}

\begin{answer}

Dit is zeker het geval. Door een gebrekkig beleid overstijgt de stroom van
toeristen overstijgt de toeristische draagkracht. Het streven van de individuele
toerist leidt tot de overexploitatie van het gemeenschappelijke goed dat het
toeristisch efgoed is. Het gevolg zijn een hoop negatieve externaliteten zoals
vervuiling, microcriminaliteit, congestie, stijgende kosten, verlies van
identiteit, etc. Het toerisme in Venetië is allesbehalve duurzaam.

\end{answer}

\section{Total Workplace Innovation}

\begin{question}
Leg uit waarom de traditionele, functionele manier van werken niet meer
aangepast is aan de omgevingseisen anno 2015.
\end{question}

\begin{answer}

\end{answer}

\section{Nieuwe Kankertherapieën}

\begin{question}
Wat is de rol van preventie en screening voor kanker en waarom zijn ze
belangrijk?
\end{question}

\begin{answer}

\end{answer}

\section{Wetenschappelijke fraude}

\begin{question}
Zijn schendingen van wetenschappelijke integriteit in alle disciplines even
frequent volgens u of waar zouden ze frequenter zijn en waarom?
\end{question}

\begin{answer}

\end{answer}

\end{document}
